%-------------------------------------------------------------------------------------------------------------------- 
%                                       Basic stuff
%-------------------------------------------------------------------------------------------------------------------- 

% do not show the box of framebox, somehow frambox fixed the bug of dir-trees not being next to each other
% \setlength{\fboxrule}{0pt}

% adjust the size of \chapter{}
\makeatletter
\renewcommand{\@makechapterhead}[1]{%
\vspace*{0 pt}%
{\setlength{\parindent}{0pt} \raggedright \normalfont
\bfseries\LARGE
\ifnum \value{secnumdepth}>1 
    \if@mainmatter\thechapter.\ \fi%
\fi
#1\par\nobreak\vspace{10 pt}}}
\makeatother

%% Right vertical brace 
% \newenvironment{rcases}
%   {\left.\begin{aligned}}
%   {\end{aligned}\right\rbrace}

% define a command for chemical reactions
\newcommand*\ch[1]{\ensuremath{\mathrm{#1}}}

%create a multiline comment by using \mycomment{ ... }
\newcommand{\mycomment}[1]{}     

%--------------------------------------------------------------------------------------------------------------------
%                                       inline code background highlighting
%--------------------------------------------------------------------------------------------------------------------
\usepackage{soul}
\newcommand{\inlcode}[1]{%
  \begingroup
  \definecolor{mygray}{rgb}{0.9,0.9,0.9}\sethlcolor{mygray}%
  \hl{#1}%
  \endgroup
}

%-------------------------------------------------------------------------------------------------------------------- 
%                                       listings for code inclusion
%-------------------------------------------------------------------------------------------------------------------- 
% \begingroup
% \let\oldnumberline\numberline
% \renewcommand{\numberline}[1]{\hspace{-1.5em}Code \oldnumberline{#1:}}
% \lstlistoflistings
% \endgroup
\makeatletter
\renewcommand{\l@lstlisting}[2]{%
  \@dottedtocline{1}{0em}{1.5em}{\lstlistingname\ #1}{#2}%
}
\makeatother

% define a skip option for the line numbers in lstlisting with supress and reactivate
\let\origthelstnumber\thelstnumber
\makeatletter
\newcommand*\Suppressnumber{%
  \lst@AddToHook{OnNewLine}{%
    \let\thelstnumber\relax%
     \advance\c@lstnumber-\@ne\relax%
    }%
}

\newcommand*\Reactivatenumber[1]{%
  \setcounter{lstnumber}{\numexpr#1-1\relax}
  \lst@AddToHook{OnNewLine}{%
   \let\thelstnumber\origthelstnumber%
   \refstepcounter{lstnumber}%
  }%
}
\makeatother

%-------------------------------------------------------------------------------------------------------------------- 
%                                       glossary styles for acronyms & symbols
%-------------------------------------------------------------------------------------------------------------------- 
% used for acronyms
\newglossarystyle{mystyle}{%
    \setglossarystyle{long}%   
    %\setlength\LTleft{0pt}%
    \renewenvironment{theglossary}%
    {\begin{longtable}[]{@{}>{\raggedright}p{0.3\textwidth}<{\dotfill}@{}>{\hspace{-0.44em}\dotfill\raggedleft}p{0.7\textwidth}@{}}} %@{}to remove padding spaces between colums
    {\end{longtable}}%
    }

% \setlength{\glsdescwidth}{18cm} %15cm
\newglossary[slg]{symbolslist}{syi}{syg}{List of Symbols} % create additional symbols list
\glsaddkey{unit}{\glsentrytext{\glslabel}}{\glsentryunit}{\GLsentryunit}{\glsunit}{\Glsunit}{\GLSunit}
\glssetnoexpandfield{unit}

\newglossarystyle{symbunitlong}{%
    \setglossarystyle{long3col}% base this style on the list style
    %\setlength{\glsdescwidth}{\linewidth}%
    \renewenvironment{theglossary}{% Change the table type --> 3 columns
        \settowidth{\dimen0}{\bfseries Superscripts}% previously Symbols but superscripts as heading is longer
        \settowidth{\dimen1}{\bfseries \si{\per\second}  $\wedge$ \si{\kelvin\tothe{\mathrm{\beta}}\per\second}}%Unit % use the longest unit definition here
        % Adjust the width of the description column
        \begin{longtable}{@{}l p{\dimexpr\linewidth-\dimen0-\dimen1}l@{}}}%
        {\end{longtable}}%
    \renewcommand*{\glossaryheader}{%  Change the table header
        \bfseries & \bfseries Description & \bfseries Unit \\\hline
        \endhead}
    \renewcommand*{\glossentry}[2]{%  Change the displayed items
        \glstarget{##1}{\glossentryname{##1}} %
        & \glossentrydesc{##1}% Description
        & \glsunit{##1}  \tabularnewline
    }%
    % use groupheadings with the sort={}
    \renewcommand*{\glsgroupheading}[1]{\tabularnewline\textbf{\glsgetgrouptitle{##1}\bigskip}\\}%
    \newcommand*{\Agroupname}{Symbols}
    \newcommand*{\Sgroupname}{Subscripts}
    \newcommand*{\Hgroupname}{Superscripts}
    }