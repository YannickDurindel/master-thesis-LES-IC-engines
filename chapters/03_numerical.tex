This chapter describes the numerical methods employed in this work, with particular focus on the in-house solver PsiPhi used for both \gls{dns} and \gls{les} simulations. The discretization schemes, time integration methods, and implementation of the wall treatment are presented.

\section{The PsiPhi Solver}

PsiPhi is an in-house computational fluid dynamics solver developed for the simulation of turbulent reacting flows. The solver is designed for high-performance computing environments and supports both \gls{dns} and \gls{les} approaches.

\subsection{Governing Equations}

PsiPhi solves the incompressible Navier-Stokes equations in their filtered form for \gls{les} or unfiltered form for \gls{dns}:
\begin{align}
    \frac{\partial u_i}{\partial x_i} &= 0 \label{eq:psiphi_cont}\\
    \frac{\partial u_i}{\partial t} + \frac{\partial (u_i u_j)}{\partial x_j} &= -\frac{1}{\rho}\frac{\partial p}{\partial x_i} + \frac{\partial}{\partial x_j}\left[(\nu + \nu_t)\left(\frac{\partial u_i}{\partial x_j} + \frac{\partial u_j}{\partial x_i}\right)\right] + f_i
    \label{eq:psiphi_mom}
\end{align}
where $\nu_t$ represents the turbulent viscosity, which is zero for \gls{dns} and computed from an \gls{sgs} model for \gls{les}.

\section{Spatial Discretization}

\subsection{Finite Volume Method}

PsiPhi employs the \gls{fvm} for spatial discretization. The computational domain is divided into control volumes, and the governing equations are integrated over each control volume. For a general transport equation:
\begin{equation}
    \frac{\partial \phi}{\partial t} + \nabla \cdot (\mathbf{u}\phi) = \nabla \cdot (\Gamma \nabla \phi) + S_\phi
    \label{eq:general_transport}
\end{equation}
the integration over a control volume $V$ with surface $A$ yields:
\begin{equation}
    \frac{\partial}{\partial t}\int_V \phi \, dV + \oint_A (\mathbf{u}\phi) \cdot d\mathbf{A} = \oint_A (\Gamma \nabla \phi) \cdot d\mathbf{A} + \int_V S_\phi \, dV
    \label{eq:fvm_integral}
\end{equation}

\subsection{Grid Arrangement}

The solver uses a staggered grid arrangement where:
\begin{itemize}
    \item Scalar quantities (pressure, temperature) are stored at cell centers
    \item Velocity components are stored at cell faces
\end{itemize}
This arrangement prevents the checkerboard pressure oscillations that can occur with collocated grids and ensures strong pressure-velocity coupling.

\subsection{Convective Term Discretization}

The convective fluxes are discretized using a combination of schemes depending on the application:

\subsubsection{Central Differencing Scheme}

For \gls{dns} and well-resolved \gls{les}, the central differencing scheme (CDS) provides second-order accuracy:
\begin{equation}
    \phi_f = \frac{\phi_P + \phi_N}{2}
    \label{eq:cds}
\end{equation}
where $\phi_f$ is the face value, and $\phi_P$ and $\phi_N$ are the values at the neighboring cell centers.

\subsubsection{Upwind Scheme}

For stability in convection-dominated flows, the upwind scheme can be employed:
\begin{equation}
    \phi_f = \begin{cases}
        \phi_P & \text{if } (\mathbf{u} \cdot \mathbf{n})_f > 0 \\
        \phi_N & \text{if } (\mathbf{u} \cdot \mathbf{n})_f < 0
    \end{cases}
    \label{eq:upwind}
\end{equation}

\subsubsection{TVD Schemes}

Total Variation Diminishing (TVD) schemes provide a balance between accuracy and stability by using flux limiters. The face value is computed as:
\begin{equation}
    \phi_f = \phi_P + \frac{1}{2}\psi(r)(\phi_N - \phi_P)
    \label{eq:tvd}
\end{equation}
where $\psi(r)$ is a limiter function and $r$ is the ratio of consecutive gradients.

\subsection{Diffusive Term Discretization}

The diffusive fluxes are discretized using central differencing:
\begin{equation}
    \left(\Gamma \frac{\partial \phi}{\partial x}\right)_f = \Gamma_f \frac{\phi_N - \phi_P}{\Delta x}
    \label{eq:diffusion_disc}
\end{equation}
For non-orthogonal grids, correction terms are added to maintain accuracy.

\section{Pressure-Velocity Coupling}

\subsection{Projection Method}

PsiPhi uses a fractional step (projection) method to enforce the incompressibility constraint. The algorithm proceeds as follows:

\textbf{Step 1: Predictor step}

Compute an intermediate velocity $u_i^*$ by solving the momentum equation without the pressure gradient:
\begin{equation}
    \frac{u_i^* - u_i^n}{\Delta t} = -\frac{\partial (u_i u_j)^n}{\partial x_j} + \nu \frac{\partial^2 u_i^n}{\partial x_j \partial x_j} + f_i
    \label{eq:predictor}
\end{equation}

\textbf{Step 2: Pressure Poisson equation}

The pressure is obtained by solving a Poisson equation derived from the continuity constraint:
\begin{equation}
    \nabla^2 p^{n+1} = \frac{\rho}{\Delta t}\nabla \cdot \mathbf{u}^*
    \label{eq:pressure_poisson}
\end{equation}

\textbf{Step 3: Corrector step}

The velocity is corrected to satisfy continuity:
\begin{equation}
    u_i^{n+1} = u_i^* - \frac{\Delta t}{\rho}\frac{\partial p^{n+1}}{\partial x_i}
    \label{eq:corrector}
\end{equation}

\subsection{Poisson Solver}

The pressure Poisson equation is solved using iterative methods. PsiPhi supports:
\begin{itemize}
    \item Jacobi iteration
    \item Gauss-Seidel iteration
    \item Successive Over-Relaxation (SOR)
    \item Conjugate Gradient methods
    \item Multigrid acceleration
\end{itemize}

\section{Temporal Discretization}

\subsection{Explicit Time Integration}

For the convective and source terms, explicit time integration schemes are employed:

\subsubsection{Euler Method}

The first-order Euler method:
\begin{equation}
    \phi^{n+1} = \phi^n + \Delta t \cdot F(\phi^n)
    \label{eq:euler}
\end{equation}

\subsubsection{Runge-Kutta Methods}

Higher-order accuracy is achieved using Runge-Kutta schemes. The third-order Runge-Kutta (RK3) method:
\begin{align}
    \phi^{(1)} &= \phi^n + \Delta t \cdot F(\phi^n) \\
    \phi^{(2)} &= \frac{3}{4}\phi^n + \frac{1}{4}\phi^{(1)} + \frac{1}{4}\Delta t \cdot F(\phi^{(1)}) \\
    \phi^{n+1} &= \frac{1}{3}\phi^n + \frac{2}{3}\phi^{(2)} + \frac{2}{3}\Delta t \cdot F(\phi^{(2)})
    \label{eq:rk3}
\end{align}

\subsection{Time Step Restriction}

The time step is limited by stability constraints:

\subsubsection{CFL Condition}

The Courant-Friedrichs-Lewy (CFL) condition for convective stability:
\begin{equation}
    \Delta t \leq \text{CFL} \cdot \min\left(\frac{\Delta x}{|u|}, \frac{\Delta y}{|v|}, \frac{\Delta z}{|w|}\right)
    \label{eq:cfl}
\end{equation}
where CFL $\leq 1$ for explicit schemes.

\subsubsection{Viscous Stability}

The diffusive stability constraint:
\begin{equation}
    \Delta t \leq \frac{\Delta x^2}{2\nu}
    \label{eq:viscous_stability}
\end{equation}

\section{Boundary Conditions}

\subsection{Inlet Boundary}

At inlet boundaries, velocity components are prescribed:
\begin{equation}
    u_i = u_{i,\text{inlet}}
    \label{eq:inlet_bc}
\end{equation}
For turbulent inflow, synthetic turbulence generators or recycling methods can be employed.

\subsection{Outlet Boundary}

Convective outflow conditions allow disturbances to exit the domain:
\begin{equation}
    \frac{\partial \phi}{\partial t} + U_c \frac{\partial \phi}{\partial n} = 0
    \label{eq:convective_outlet}
\end{equation}
where $U_c$ is the convection velocity and $n$ is the outward normal direction.

\subsection{Wall Boundary}

At solid walls, the no-slip condition is applied:
\begin{equation}
    u_i = 0
    \label{eq:no_slip}
\end{equation}

For the pressure, a zero normal gradient condition is used:
\begin{equation}
    \frac{\partial p}{\partial n} = 0
    \label{eq:pressure_wall}
\end{equation}

\subsection{Periodic Boundary}

For channel flow simulations, periodic boundary conditions are applied in the streamwise and spanwise directions:
\begin{equation}
    \phi(x) = \phi(x + L_x), \quad \phi(z) = \phi(z + L_z)
    \label{eq:periodic}
\end{equation}

\section{Immersed Boundary Implementation}

PsiPhi uses an immersed boundary method to represent complex geometries on the Cartesian grid. This section describes the implementation details.

\subsection{Geometry Representation}

The solid geometry is represented using a signed distance function $\phi(\mathbf{x})$:
\begin{equation}
    \phi(\mathbf{x}) = \begin{cases}
        > 0 & \text{fluid region} \\
        = 0 & \text{boundary surface} \\
        < 0 & \text{solid region}
    \end{cases}
    \label{eq:signed_distance}
\end{equation}

The distance function is computed from the CAD geometry (typically STL format) and stored on the grid. For moving boundaries, the distance function is updated at each time step.

\subsection{Cell Classification}

Grid cells are classified based on the signed distance function:
\begin{itemize}
    \item \textbf{Fluid cells}: Entirely in the fluid region ($\phi > 0$ at all vertices)
    \item \textbf{Solid cells}: Entirely in the solid region ($\phi < 0$ at all vertices)
    \item \textbf{Cut cells}: Intersected by the boundary (mixed signs at vertices)
\end{itemize}

Cut cells require special treatment to enforce boundary conditions while maintaining conservation.

\subsection{Velocity Boundary Conditions}

The no-slip condition at immersed boundaries is enforced through direct forcing. For a cut cell, the velocity is interpolated to satisfy:
\begin{equation}
    u_{IB} = u_{wall} + (u_{fluid} - u_{wall}) \cdot f(\phi)
    \label{eq:ib_velocity}
\end{equation}
where $f(\phi)$ is a blending function based on the distance to the wall.

For moving boundaries such as the piston, the wall velocity $u_{wall}$ corresponds to the local surface velocity.

\subsection{Pressure Treatment}

The pressure in solid cells is extrapolated from the fluid region to ensure smooth pressure gradients across the immersed boundary:
\begin{equation}
    \frac{\partial p}{\partial n}\bigg|_{wall} = 0
    \label{eq:ib_pressure}
\end{equation}

This Neumann condition is consistent with the no-penetration constraint at solid walls.

\section{LES Implementation}

\subsection{Subgrid-Scale Model}

For \gls{les}, PsiPhi implements the Smagorinsky model with the \gls{sgs} viscosity:
\begin{equation}
    \nu_{sgs} = (C_s \Delta)^2 |\tilde{S}|
    \label{eq:sgs_psiphi}
\end{equation}
where $\Delta = (\Delta x \cdot \Delta y \cdot \Delta z)^{1/3}$ is the filter width based on the local grid spacing.

\subsection{Near-Wall Treatment}

Near solid walls, the Smagorinsky model requires damping to account for the reduced turbulent length scales. The van Driest damping function is applied:
\begin{equation}
    f_d = 1 - \exp\left(-\frac{y^+}{A^+}\right)
    \label{eq:van_driest}
\end{equation}
where $A^+ \approx 25$ is the damping constant.

\section{Wall Treatment: Enhanced Molecular Viscosity Approach}

The primary focus of this work is the development of a novel wall treatment for \gls{les} that accounts for boundary-layer effects by introducing a locally increased molecular viscosity near the walls. This approach is specifically designed for use with immersed boundaries.

\subsection{Motivation}

Classical wall functions fail under the rapidly changing and non-equilibrium conditions of internal combustion engines. The enhanced molecular viscosity approach offers several advantages:
\begin{itemize}
    \item Does not assume equilibrium boundary layer profiles
    \item Compatible with immersed boundary methods
    \item Computationally efficient for complex moving geometries
    \item Can be calibrated and verified against standard wall functions
\end{itemize}

\subsection{Formulation}

The effective viscosity in the near-wall region is modified to account for unresolved turbulent transport:
\begin{equation}
    \nu_{eff}(y) = \nu \cdot \left(1 + \alpha \cdot f(y/\delta)\right)
    \label{eq:enhanced_mol_visc}
\end{equation}
where:
\begin{itemize}
    \item $\nu$ is the molecular (kinematic) viscosity
    \item $\alpha$ is the enhancement factor (to be determined)
    \item $f(y/\delta)$ is a profile function depending on wall distance
    \item $\delta$ is a characteristic length scale (boundary layer thickness or cell size)
\end{itemize}

The key distinction from turbulent viscosity models is that this approach directly manipulates the molecular viscosity coefficient, which governs the diffusive fluxes in the momentum equation.

\subsection{1D Model Development}

A one-dimensional model is developed to determine the viscosity profile across the boundary layer. The 1D model solves a simplified momentum balance in the wall-normal direction:
\begin{equation}
    \frac{d}{dy}\left(\nu_{eff}(y) \frac{du}{dy}\right) = \frac{1}{\rho}\frac{dp}{dx}
    \label{eq:1d_model}
\end{equation}

This equation is integrated from the wall to the edge of the boundary layer to obtain the velocity profile and wall shear stress. The viscosity profile $\nu_{eff}(y)$ is adjusted to match the law of the wall behavior.

\subsection{Verification Strategy}

The model is verified using standard wall functions. For equilibrium boundary layers, the enhanced viscosity model should reproduce:
\begin{itemize}
    \item Linear velocity profile in the viscous sublayer: $u^+ = y^+$
    \item Logarithmic profile in the log layer: $u^+ = \frac{1}{\kappa}\ln(y^+) + B$
    \item Correct wall shear stress: $\tau_w = \mu \frac{\partial u}{\partial y}\bigg|_{wall}$
\end{itemize}

The verification ensures that the model recovers classical results before application to non-equilibrium engine flows.

\subsection{Implementation in PsiPhi}

The enhanced viscosity is implemented by modifying the diffusive term in the momentum equation for cells near immersed boundaries:

\textbf{Step 1: Wall Distance Computation}

For each cell near an immersed boundary, the wall distance $y$ is computed from the signed distance function.

\textbf{Step 2: Viscosity Enhancement}

The local viscosity is modified according to:
\begin{equation}
    \nu_{cell} = \nu \cdot \left(1 + \alpha \cdot f\left(\frac{y}{\Delta}\right)\right)
    \label{eq:implementation_visc}
\end{equation}
where $\Delta$ is the local grid spacing.

\textbf{Step 3: Blending}

A smooth transition to the standard \gls{les} viscosity is applied:
\begin{equation}
    \nu_{total} = \nu_{cell} + \nu_{sgs}
    \label{eq:total_visc}
\end{equation}

The blending function $f$ approaches zero far from the wall, ensuring that the standard \gls{les} behavior is recovered in the bulk flow.

\subsection{Coupling with Immersed Boundaries}

The enhanced viscosity wall treatment is naturally coupled with the immersed boundary method:
\begin{itemize}
    \item The signed distance function provides the wall distance
    \item Cut cells receive the enhanced viscosity based on their distance to the surface
    \item Moving boundaries automatically update the wall distance
    \item No special grid requirements near the wall
\end{itemize}

This coupling enables consistent treatment of all solid surfaces in the engine, including the moving piston.

\section{Channel Flow Configuration}

\subsection{Computational Domain}

The channel flow simulations use a domain of size $L_x \times L_y \times L_z$ with:
\begin{itemize}
    \item $L_x = 2\pi h$ in the streamwise direction
    \item $L_y = 2h$ in the wall-normal direction (channel half-height $h$)
    \item $L_z = \pi h$ in the spanwise direction
\end{itemize}

\subsection{Grid Requirements}

For \gls{dns}, the grid resolution must satisfy:
\begin{itemize}
    \item $\Delta x^+ \approx 10$--$15$ (streamwise)
    \item $\Delta y^+_{wall} < 1$ (first cell at wall)
    \item $\Delta z^+ \approx 5$--$10$ (spanwise)
\end{itemize}

For \gls{les} with wall treatment:
\begin{itemize}
    \item $\Delta x^+ \approx 50$--$100$ (streamwise)
    \item $\Delta y^+_{wall} \approx 1$--$30$ depending on wall model
    \item $\Delta z^+ \approx 20$--$50$ (spanwise)
\end{itemize}

\subsection{Flow Driving}

The channel flow is driven by a constant pressure gradient in the streamwise direction. The pressure gradient is related to the target friction Reynolds number:
\begin{equation}
    \frac{dp}{dx} = -\frac{\rho u_\tau^2}{h} = -\frac{\rho \nu^2 Re_\tau^2}{h^3}
    \label{eq:pressure_gradient}
\end{equation}

\section{Post-Processing and Statistics}

\subsection{Temporal Averaging}

Statistical quantities are computed by time-averaging over a sufficiently long period after the flow has reached a statistically steady state:
\begin{equation}
    \langle \phi \rangle = \frac{1}{T}\int_0^T \phi(t) \, dt
    \label{eq:time_avg}
\end{equation}

\subsection{Spatial Averaging}

For channel flow, additional averaging is performed over the homogeneous directions (streamwise and spanwise):
\begin{equation}
    \langle \phi \rangle_{xz} = \frac{1}{L_x L_z}\int_0^{L_x}\int_0^{L_z} \phi(x,y,z) \, dx \, dz
    \label{eq:spatial_avg}
\end{equation}

\subsection{Computed Quantities}

The following quantities are extracted for comparison:
\begin{itemize}
    \item Mean velocity: $\langle u \rangle^+ = \langle u \rangle / u_\tau$
    \item Velocity fluctuations: $u_{rms}^+ = \sqrt{\langle u'^2 \rangle} / u_\tau$
    \item Reynolds shear stress: $\langle u'v' \rangle^+ = \langle u'v' \rangle / u_\tau^2$
    \item Turbulent kinetic energy: $k^+ = k / u_\tau^2$
\end{itemize}
