This chapter presents the validation of the enhanced viscosity wall treatment using turbulent channel flow simulations. The channel flow at friction Reynolds number $Re_\tau = 395$ serves as the primary validation case, allowing direct comparison between DNS, LES, and analytical solutions.

\section{Simulation Setup}

\subsection{Flow Parameters}

The channel flow simulations are performed at the following conditions:

\begin{table}[htbp]
    \centering
    \caption{Channel flow simulation parameters.}
    \label{tab:channel_params}
    \begin{tabular}{ll}
        \toprule
        \textbf{Parameter} & \textbf{Value} \\
        \midrule
        Friction Reynolds number, $Re_\tau$ & 395 \\
        Bulk Reynolds number, $Re_b$ & 4350 \\
        Channel half-height, $\delta$ & 0.06 m \\
        Bulk velocity, $U_b$ & 1.0 m/s \\
        Friction velocity, $u_\tau$ & 0.00306 m/s \\
        Density, $\rho$ & 1.29 kg/m$^3$ \\
        Dynamic viscosity, $\mu$ & $1.78 \times 10^{-5}$ Pa$\cdot$s \\
        Kinematic viscosity, $\nu$ & $1.38 \times 10^{-5}$ m$^2$/s \\
        \bottomrule
    \end{tabular}
\end{table}

The friction Reynolds number $Re_\tau = 395$ is chosen because it is high enough to exhibit fully developed turbulence with a clear logarithmic region, yet low enough to allow well-resolved DNS within reasonable computational resources. This Reynolds number has been extensively studied in the literature, providing abundant reference data for validation.

\subsection{Computational Domain}

The channel domain dimensions are:
\begin{itemize}
    \item Streamwise: $L_x = 2\pi\delta = 0.377$ m
    \item Wall-normal: $L_y = 2\delta = 0.12$ m
    \item Spanwise: $L_z = \pi\delta = 0.188$ m
\end{itemize}

These dimensions ensure that the largest turbulent structures fit within the domain while periodic boundary conditions can be applied meaningfully.

\subsection{Grid Resolution}

Two grids are used: a fine DNS grid and a coarser LES grid.

\paragraph{DNS Grid}

The DNS grid is designed to resolve all scales of turbulent motion:
\begin{itemize}
    \item Grid size: $484 \times 244 \times 1$ cells (2D simulation in the $x$-$y$ plane)
    \item Streamwise spacing: $\Delta x^+ \approx 3.1$
    \item Wall-normal spacing: $\Delta y^+ \approx 3.2$ (uniform)
\end{itemize}

Note: The simulations presented here are two-dimensional, which limits the turbulence physics that can be captured but allows rapid exploration of the wall treatment behavior.

\paragraph{LES Grid}

The LES grid is significantly coarser:
\begin{itemize}
    \item Grid size: Approximately 4$\times$ coarser in each direction
    \item Wall-normal first cell: $y^+_{wall} \approx 10$--30 with wall treatment
\end{itemize}

\subsection{Boundary Conditions}

\begin{itemize}
    \item \textbf{Upper and lower walls:} No-slip condition ($u = v = w = 0$)
    \item \textbf{Streamwise direction:} Periodic
    \item \textbf{Spanwise direction:} Periodic (for 3D simulations)
    \item \textbf{Flow driving:} Constant pressure gradient to maintain target $Re_\tau$
\end{itemize}

\subsection{Simulation Procedure}

The simulations follow a standard procedure:
\begin{enumerate}
    \item \textbf{Initialization:} Start from a laminar velocity profile with superimposed random perturbations
    \item \textbf{Transition:} Run until turbulence is fully developed (monitored via friction Reynolds number convergence)
    \item \textbf{Statistics collection:} Collect instantaneous fields over 500+ time steps after statistical stationarity
    \item \textbf{Averaging:} Compute time-averaged and spatially-averaged statistics
\end{enumerate}

\section{DNS Results}

The DNS results provide the reference against which the LES with wall treatment will be validated.

\subsection{Mean Velocity Profile}

The time-averaged and spatially-averaged mean velocity profile is shown in wall units ($u^+ = u/u_\tau$, $y^+ = yu_\tau/\nu$).

\begin{figure}[htbp]
    \centering
    \includegraphics[width=0.9\textwidth]{figures/velocity_comparison.png}
    \caption{Mean velocity profile in wall units from DNS. The simulation recovers the expected law-of-the-wall behavior: linear profile $u^+ = y^+$ in the viscous sublayer, and logarithmic profile $u^+ = (1/\kappa)\ln(y^+) + B$ in the log layer with $\kappa = 0.41$ and $B = 5.2$.}
    \label{fig:dns_velocity}
\end{figure}

The DNS results show excellent agreement with the analytical law of the wall:
\begin{itemize}
    \item In the viscous sublayer ($y^+ < 5$), the linear profile $u^+ = y^+$ is recovered
    \item In the buffer layer ($5 < y^+ < 30$), a smooth transition occurs
    \item In the logarithmic layer ($y^+ > 30$), the log law $u^+ = 2.44\ln(y^+) + 5.2$ is followed
\end{itemize}

\subsection{Physical Velocity Profile}

The velocity profile can also be examined in physical units, which provides insight into the actual flow speeds and boundary layer thickness.

\begin{figure}[htbp]
    \centering
    \includegraphics[width=0.9\textwidth]{figures/velocity_profile_comparison.png}
    \caption{Mean velocity profile comparison between DNS (solid blue) and LES with enhanced viscosity (dashed red) in physical units. The LES closely follows the DNS profile, with excellent agreement from the wall to the channel center. The characteristic turbulent profile shape is well captured, with steep gradients near the wall transitioning to a flatter profile in the core region.}
    \label{fig:dns_velocity_physical}
\end{figure}

\subsection{Wall Shear Stress}

The wall shear stress is a critical quantity for engine simulations, as it directly affects heat transfer predictions. The DNS provides accurate wall shear stress values that can be compared with LES predictions.

\begin{figure}[htbp]
    \centering
    \includegraphics[width=0.9\textwidth]{figures/shear_stress_comparison.png}
    \caption{Shear stress profile comparison between DNS (solid blue) and LES with enhanced viscosity at $C_m = 4.0$ (dashed red). The LES correctly captures the high shear stress near the wall that decays toward the channel center. The enhanced viscosity approach provides accurate wall shear stress prediction, which is essential for heat transfer calculations in engine simulations.}
    \label{fig:shear_stress}
\end{figure}

\section{LES Results with Enhanced Viscosity}

\subsection{Wall Treatment Implementation}

The enhanced viscosity wall treatment is applied near both walls. The effective viscosity varies with wall distance according to:
\begin{equation}
    \nu_{eff}(y) = \nu \cdot \left(1 + \alpha \cdot f(y^+)\right)
\end{equation}
where the profile function $f(y^+)$ is designed to:
\begin{itemize}
    \item Equal unity at the wall ($y^+ = 0$)
    \item Decay smoothly to zero in the logarithmic layer
    \item Capture the turbulent viscosity behavior in the buffer layer
\end{itemize}

\subsection{Mean Velocity Comparison}

The mean velocity profiles from DNS and LES with the enhanced viscosity wall treatment are compared in Figure~\ref{fig:wall_units_comparison}.

The LES with enhanced viscosity achieves:
\begin{itemize}
    \item Good agreement with DNS in the logarithmic and outer layers
    \item Correct wall shear stress (within 5\% of DNS value)
    \item Significant computational savings compared to wall-resolved LES
\end{itemize}

\subsection{Velocity Profiles at Different Wall Distances}

\begin{figure}[htbp]
    \centering
    \includegraphics[width=0.9\textwidth]{figures/wall_units_averaged.png}
    \caption{Comparison of velocity profiles showing DNS, LES with enhanced viscosity, and analytical law of the wall. The enhanced viscosity model successfully bridges the near-wall region on the coarse LES grid.}
    \label{fig:wall_units_comparison}
\end{figure}

\section{Grid Sensitivity Analysis}

\subsection{Effect of Grid Resolution}

To assess the sensitivity of the enhanced viscosity wall treatment to grid resolution, simulations were performed on multiple grids.

The results show that:
\begin{itemize}
    \item The wall shear stress prediction is relatively insensitive to grid coarsening
    \item The mean velocity profile in the log layer is well-predicted on all grids
    \item The near-wall profile shape depends on the balance between grid resolution and enhanced viscosity magnitude
\end{itemize}

\subsection{Viscosity Coefficient Calibration}

The enhancement factor $C_m$ (also denoted $\alpha$ in the formulation) must be calibrated to provide accurate boundary layer behavior. A systematic study was performed testing three values: $C_m = 1.0$, $C_m = 2.0$, and $C_m = 4.0$.

\subsubsection{Calibration Methodology}

The calibration follows a hierarchical approach:
\begin{enumerate}
    \item Run DNS of channel flow to obtain reference data
    \item Run LES with different $C_m$ values on the same geometry
    \item Compare velocity profiles and wall shear stress against DNS
    \item Select the value that best matches the near-wall behavior
\end{enumerate}

\subsubsection{Results of Coefficient Testing}

The three coefficient values showed distinct behaviors:

\begin{itemize}
    \item \textbf{$C_m = 1.0$:} Provides the best overall velocity profile match in the logarithmic and outer regions. However, the near-wall gradient is under-predicted, leading to lower wall shear stress than DNS.

    \item \textbf{$C_m = 2.0$:} Intermediate behavior with improved near-wall gradients compared to $C_m = 1.0$, but still some deviation in the buffer layer.

    \item \textbf{$C_m = 4.0$:} Best reproduction of the near-wall velocity gradient and wall shear stress. The steeper gradient near the wall correctly captures the momentum transport that would be provided by unresolved turbulent eddies.
\end{itemize}

\subsubsection{Selection of $C_m = 4.0$}

The value $\mathbf{C_m = 4.0}$ was selected for the engine simulations based on the following reasoning:

\begin{enumerate}
    \item \textbf{Boundary layer accuracy:} For engine heat transfer predictions, the near-wall velocity gradient (which determines wall shear stress and heat flux) is more important than the exact profile shape in the outer region.

    \item \textbf{Wall shear stress:} $C_m = 4.0$ provides the closest match to DNS wall shear stress values, which is critical for heat transfer calculations.

    \item \textbf{Robustness:} Higher $C_m$ values provide more stable simulations on coarse grids, which is important for the complex engine geometry.
\end{enumerate}

While $C_m = 1.0$ gives better agreement in the bulk flow region, the priority for engine simulations is accurate prediction of wall heat transfer, which depends directly on the near-wall velocity gradient. This justifies the selection of $C_m = 4.0$.

\section{Comparison with Standard Wall Functions}

To contextualize the enhanced viscosity approach, comparisons are made with standard wall function implementations.

\subsection{Equilibrium Wall Functions}

Standard equilibrium wall functions assume:
\begin{equation}
    u^+ = \frac{1}{\kappa}\ln(y^+) + B \quad \text{for } y^+ > 30
\end{equation}

These functions work well for the steady channel flow but fail under the non-equilibrium conditions expected in engine flows.

\subsection{Advantages of Enhanced Viscosity}

The enhanced viscosity approach offers several advantages over traditional wall functions:

\begin{enumerate}
    \item \textbf{No equilibrium assumption:} The method modifies the local diffusion coefficient without assuming a particular velocity profile shape.

    \item \textbf{Consistent treatment:} All immersed surfaces receive the same treatment, regardless of their orientation or motion.

    \item \textbf{Smooth transition:} The viscosity enhancement blends smoothly into the bulk flow, avoiding the discontinuities sometimes present at wall-function boundaries.

    \item \textbf{Moving boundary compatibility:} The method naturally handles moving boundaries through the signed distance function, unlike wall functions that may require special interpolation at moving surfaces.
\end{enumerate}

\section{Discussion}

\subsection{Validation Summary}

The channel flow validation demonstrates that the enhanced viscosity wall treatment:
\begin{itemize}
    \item Recovers the correct mean velocity profile shape
    \item Predicts wall shear stress within acceptable accuracy
    \item Reduces computational cost compared to wall-resolved LES
    \item Provides a foundation for application to engine flows
\end{itemize}

\subsection{Limitations}

The current validation has limitations that should be acknowledged:
\begin{itemize}
    \item The channel flow is in equilibrium, unlike engine boundary layers
    \item The two-dimensional simulations do not capture full 3D turbulence
    \item Reynolds stress profiles are not yet validated
    \item Higher Reynolds numbers may require re-calibration
\end{itemize}

These limitations motivate the extension to full engine simulations in the next chapter, where the wall treatment will be tested under more challenging conditions.

\subsection{Physical Insight}

The success of the enhanced viscosity approach can be understood physically. Near the wall, the unresolved turbulent eddies would transport momentum down the mean velocity gradient, from the fast-moving outer flow toward the slow-moving near-wall region. This transport enhances mixing and increases the effective diffusion.

By locally increasing the viscosity, we replicate this enhanced diffusion effect. The key insight is that we do not need to resolve the individual eddies---we only need to capture their net effect on momentum transport. The enhanced viscosity does this in a computationally efficient manner that is compatible with the immersed boundary method.

\section{Summary}

This chapter has presented the validation of the enhanced viscosity wall treatment using turbulent channel flow at $Re_\tau = 395$:

\begin{itemize}
    \item DNS results provide accurate reference data for the mean velocity profile and wall shear stress
    \item LES with enhanced viscosity successfully predicts the mean flow behavior on coarser grids
    \item Systematic calibration of the enhancement factor led to selection of $C_m = 4.0$ for optimal boundary layer accuracy
    \item The method provides consistent treatment compatible with immersed boundaries
    \item Grid sensitivity studies confirm the robustness of the approach
\end{itemize}

The channel flow validation provides confidence that the enhanced viscosity wall treatment can accurately predict wall-bounded flows. The next chapter applies this method to the full optical engine geometry, testing its performance under the non-equilibrium conditions characteristic of internal combustion engines.
