Wall-bounded flows are inherently complex because they give rise to boundary layers. Near the wall, steep velocity gradients develop due to the no-slip condition. In internal combustion engines, these layers form along multiple surfaces, with the piston being particularly significant due to its large area, continuous motion, and strong influence on large-scale flow structures such as the tumble vortex.

\section{Motivation}

\subsection{Boundary Layers in Internal Combustion Engines}

Unlike steady external flows, engine flows are highly unsteady and dominated by pressure fluctuations and non-equilibrium effects. The in-cylinder flow undergoes rapid changes during the intake, compression, combustion, and exhaust strokes, creating a complex turbulent environment where classical boundary layer assumptions often fail.

The piston surface plays a critical role in engine flow dynamics:
\begin{itemize}
    \item Large surface area exposed to combustion gases
    \item Continuous motion affecting the boundary layer development
    \item Strong influence on the tumble vortex and large-scale flow structures
\end{itemize}

Accurate prediction of wall shear stress at the piston surface is essential for:
\begin{itemize}
    \item Understanding in-cylinder mixing and flow patterns
    \item Predicting boundary layer behavior during combustion
    \item Optimizing combustion chamber design
\end{itemize}

\subsection{Limitations of Classical Wall Functions}

Classical boundary-layer theory, developed mainly for steady conditions, describes how shear stress evolves with distance from the wall through three distinct regions: the viscous sublayer, the buffer layer, and the logarithmic region. Wall functions have been proposed to relate wall distance to shear stress across these zones, enabling coarse grid simulations.

However, the assumptions underlying these wall functions fail under the rapidly changing and confined conditions of internal combustion engines:
\begin{itemize}
    \item Non-equilibrium effects due to rapid pressure changes
    \item Strong adverse pressure gradients during compression
    \item Flow reversal and boundary layer separation
    \item Complex geometry effects near valve seats and piston bowl
\end{itemize}

Inaccurate wall shear-stress estimation leads to errors in predicting boundary layer behavior and in-cylinder flow dynamics.

\section{Challenges in Engine Simulation}

\subsection{Computational Cost}

Because of the high velocities and turbulence in engines, fully resolving all flow structures through \gls{dns} is impractical. The Reynolds numbers encountered in typical engine operation would require computational resources far beyond current capabilities.

\Gls{les} offers a practical approach by resolving the large, energy-containing eddies while modeling the smaller scales. However, \gls{les} relies on \gls{sgs} models to represent the unresolved turbulence, and the accuracy of these models near walls remains a significant challenge.

\subsection{Near-Wall Modeling Gap}

A significant knowledge gap remains in developing \gls{sgs} models that can accurately capture boundary-layer formation near the piston during the engine cycle. The standard approaches either:
\begin{itemize}
    \item Require extremely fine grids near walls (wall-resolved \gls{les}), making simulations prohibitively expensive
    \item Use wall functions that fail under engine-relevant conditions (wall-modeled \gls{les})
\end{itemize}

This motivates the development of novel wall treatment approaches that can handle the non-equilibrium conditions in engines while maintaining computational efficiency.

\section{Objectives of This Work}

This project aims to improve the physical understanding of near-wall processes in internal combustion engines and propose a novel modeling approach that accounts for boundary-layer effects. The key innovation is the introduction of a locally increased molecular viscosity near the walls, which are modeled using immersed boundaries.

The specific objectives are:
\begin{enumerate}
    \item Generation of a CAD model of the University of Duisburg-Essen optical engine, with emphasis on accurate valve seat geometry
    \item Generation of a numerical grid for the optical engine with 0.2~mm cell size
    \item Development of a 1D model incorporating molecular viscosity manipulation near walls
    \item Verification of the model using standard wall functions
    \item \gls{les} of a simplified engine configuration including the 1D model
    \item Grid sensitivity analysis to identify potential model improvements
    \item \gls{les} of the full optical engine including the wall treatment model
    \item Testing of different numerical frameworks: time integration and discretization schemes
    \item Post-processing of simulation results with focus on piston-wall phenomena
\end{enumerate}

\section{Methodology}

\subsection{Enhanced Viscosity Approach}

The proposed wall treatment introduces a locally increased molecular viscosity in the near-wall region to account for unresolved turbulent transport. Unlike traditional wall functions that prescribe velocity profiles, this approach modifies the effective viscosity to produce the correct wall shear stress on coarser grids.

The enhanced viscosity is applied within cells adjacent to the wall, where the immersed boundary method is used to represent the solid surfaces. This combination allows:
\begin{itemize}
    \item Flexible handling of complex, moving geometries (piston motion)
    \item Smooth transition between near-wall and bulk flow regions
    \item Compatibility with existing \gls{les} subgrid-scale models
\end{itemize}

\subsection{Immersed Boundary Method}

The immersed boundary method represents solid surfaces that do not conform to the computational grid. This approach is particularly suited for engine simulations where:
\begin{itemize}
    \item The piston moves through the domain during the simulation
    \item Complex valve and port geometries must be captured
    \item Grid generation for body-fitted meshes would be challenging
\end{itemize}

The wall treatment is integrated with the immersed boundary formulation to ensure consistent application of the enhanced viscosity near all solid surfaces.

\subsection{1D Model Development}

A one-dimensional model is developed to relate the enhanced viscosity to the local flow conditions and wall distance. This model:
\begin{itemize}
    \item Captures the essential physics of the near-wall region
    \item Is computationally efficient for application in three-dimensional \gls{les}
    \item Can be verified against standard wall function predictions
    \item Is calibrated to produce correct wall shear stress behavior
\end{itemize}

\subsection{Optical Engine Application}

The University of Duisburg-Essen optical engine serves as the primary application case. Optical engines provide:
\begin{itemize}
    \item Well-characterized geometry and operating conditions
    \item Experimental data for validation (when available)
    \item Representative engine flow features (tumble, swirl, squish)
    \item Realistic boundary layer development on the piston surface
\end{itemize}

\section{Focus Areas}

The post-processing and analysis focus on piston-wall phenomena:

\subsection{Shear Stress Evolution}

The wall shear stress distribution over the piston surface is analyzed throughout the engine cycle. Key aspects include:
\begin{itemize}
    \item Temporal evolution during intake, compression, and expansion
    \item Spatial distribution across the piston crown
    \item Influence of the tumble vortex on local shear stress
    \item Comparison with standard wall function predictions
\end{itemize}

\subsection{Boundary Layer Behavior}

The boundary layer structure and behavior are examined, including:
\begin{itemize}
    \item Boundary layer thickness evolution
    \item Assessment of boundary layer detachment and reattachment
    \item Non-equilibrium effects during rapid transients
    \item Interaction with large-scale flow structures
\end{itemize}

\section{Thesis Outline}

This thesis is organized as follows:

\textbf{Chapter 2: State of the Art} reviews the fundamental theory of turbulent flows and simulation methods. Conservation laws and the Navier-Stokes equations are presented, followed by the concepts of turbulence and the energy cascade. The three main simulation approaches---\gls{dns}, \gls{rans}, and \gls{les}---are described, with particular attention to \gls{les} filtering and subgrid-scale models. Wall treatment methods, immersed boundary techniques, and engine flow characteristics are introduced.

\textbf{Chapter 3: Numerical Methods} describes the PsiPhi solver used in this work. The spatial and temporal discretization schemes are presented, along with the pressure-velocity coupling algorithm. The implementation of \gls{les}, immersed boundaries, and the enhanced viscosity wall treatment are detailed.

\textbf{Chapter 4: CAD Model and Engine Geometry} presents the development of the CAD model for the optical engine used in this study. The engine geometry, based on the AVL 5811 single-cylinder research engine at the University of Duisburg-Essen, is described along with the computational domain and mesh generation process.

\textbf{Chapter 5: Channel Flow Validation} presents the validation of the enhanced viscosity wall treatment using turbulent channel flow simulations. The channel flow at friction Reynolds number $Re_\tau = 395$ serves as the primary validation case, allowing direct comparison between DNS, LES, and analytical solutions.

\textbf{Chapter 6: Engine Simulation} presents the application of the enhanced viscosity wall treatment to Large Eddy Simulation of the optical engine. Building on the validation from channel flow, the wall model is tested under the challenging non-equilibrium conditions characteristic of internal combustion engine flows.

\textbf{Chapter 7: Conclusion} summarizes the findings of this work, discusses limitations, and suggests directions for future research.
