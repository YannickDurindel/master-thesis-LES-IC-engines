Internal combustion engines remain a critical technology for transportation and power generation, with ongoing development focused on improving efficiency and reducing emissions. Accurate computational fluid dynamics (CFD) simulations are essential tools for understanding and optimizing in-cylinder flows and combustion. However, the wall-bounded turbulent flows in engines present significant modeling challenges due to their inherently non-equilibrium nature.

Wall-bounded flows near the piston, cylinder head, and liner surfaces form boundary layers that directly affect mixing and combustion efficiency. Unlike steady external flows where classical wall functions perform well, engine boundary layers experience rapid pressure changes, flow reversal, and unsteady forcing throughout the engine cycle. These non-equilibrium effects cause classical wall function assumptions---based on equilibrium boundary layer theory---to break down, potentially introducing significant errors in wall shear stress predictions.

This thesis develops and validates a novel wall treatment approach for Large Eddy Simulation (LES) of internal combustion engine flows. The enhanced viscosity wall treatment increases the effective molecular viscosity near walls to account for unresolved turbulent transport, mimicking the diffusive effect of near-wall eddies without requiring equilibrium assumptions. The approach is specifically designed for compatibility with immersed boundary methods, enabling efficient simulation of complex moving geometries such as the piston and valves.

The methodology is implemented in PsiPhi, an in-house CFD solver developed at the University of Duisburg-Essen for turbulent reacting flow simulations. PsiPhi employs structured Cartesian grids with immersed boundary methods for geometry representation, providing computational efficiency and natural handling of moving boundaries.

Validation is performed using turbulent channel flow at friction Reynolds number $Re_\tau = 395$. Direct Numerical Simulation (DNS) provides reference data for the mean velocity profile and wall shear stress. LES with the enhanced viscosity wall treatment demonstrates:
\begin{itemize}
    \item Recovery of the correct velocity profile shape in both viscous and logarithmic layers
    \item Wall shear stress predictions within 5\% of DNS values
    \item Significant computational savings compared to wall-resolved LES
    \item Robustness across a range of grid resolutions
\end{itemize}

Following validation, the enhanced viscosity wall treatment is applied to LES of the University of Duisburg-Essen optical engine under motored conditions. The optical engine provides experimental PIV data for comparison, enabling assessment of the simulation accuracy. The engine simulations capture:
\begin{itemize}
    \item Development and breakdown of the tumble vortex during intake and compression
    \item Non-equilibrium boundary layer behavior under rapid pressure changes
    \item Spatial and temporal variation of wall shear stress across the piston surface
    \item Consistent wall treatment on all surfaces including the moving piston
\end{itemize}

The computational cost of engine LES with the enhanced viscosity wall treatment is reduced by approximately an order of magnitude compared to wall-resolved approaches, making multi-cycle simulations feasible with current high-performance computing resources.

Key contributions of this work include:
\begin{enumerate}
    \item Development of an enhanced viscosity wall treatment that avoids equilibrium assumptions
    \item Demonstration of compatibility with immersed boundary methods for moving geometries
    \item Validation against DNS and analytical solutions in channel flow
    \item Application to realistic engine geometry with comparison to PIV measurements
    \item Practical demonstration of computational efficiency gains
\end{enumerate}

The results demonstrate that the enhanced viscosity wall treatment provides a viable approach for practical engine simulations, capturing the essential physics of wall-bounded flows while maintaining computational tractability. The methodology contributes to the ongoing effort to develop predictive CFD tools for internal combustion engine design and optimization.
