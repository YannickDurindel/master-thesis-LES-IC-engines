This chapter presents the CAD model development for the optical engine used in this study. The engine geometry is based on the University of Duisburg-Essen optical research engine, a facility designed to provide optical access for laser-based diagnostics while maintaining realistic engine operating conditions.

\section{The Optical Engine Facility}

\subsection{Overview}

The optical engine at the University of Duisburg-Essen is an AVL 5811 single-cylinder research engine specifically designed for fundamental studies of in-cylinder flow and combustion processes. The transparent components allow researchers to apply advanced laser-based measurement techniques such as Particle Image Velocimetry (PIV) and Laser-Induced Fluorescence (LIF) to study the flow field and species distributions within the combustion chamber.

\subsection{Engine Specifications}

The key specifications of the optical engine are summarized in Table~\ref{tab:engine_specs}.

\begin{table}[htbp]
    \centering
    \caption{Specifications of the AVL 5811 optical engine.}
    \label{tab:engine_specs}
    \begin{tabular}{ll}
        \toprule
        \textbf{Parameter} & \textbf{Value} \\
        \midrule
        Engine type & Single-cylinder, four-stroke \\
        Bore & 84 mm \\
        Stroke & 90 mm \\
        Displacement & 499 cm$^3$ \\
        Connecting rod length & 161 mm \\
        Compression ratio & $\sim$10:1 \\
        Valve configuration & 4-valve pentroof \\
        Nominal speed & 3000 rpm \\
        Optical access & Piston crown (sapphire), cylinder liner (quartz) \\
        \bottomrule
    \end{tabular}
\end{table}

\subsection{Optical Access Features}

The engine provides optical access through several transparent components:

\begin{itemize}
    \item \textbf{Extended piston with quartz window:} A Bowditch-type extended piston design allows a mirror to be placed below the piston, enabling through-piston imaging of the combustion chamber from below. The piston crown contains a large quartz window.

    \item \textbf{Quartz cylinder liner:} A section of the cylinder liner is made from fused quartz, providing lateral optical access. This enables side-view imaging and laser sheet illumination for PIV measurements.

    \item \textbf{Pentroof window:} Small windows in the pentroof provide additional access angles for laser diagnostics.
\end{itemize}

These optical access features impose constraints on the engine design---the quartz components have different thermal properties than metal, and the extended piston increases the effective crank-slider mechanism length. However, the geometric similarity to production engines is maintained as closely as possible.

\begin{figure}[htbp]
    \centering
    \includegraphics[width=0.7\textwidth]{figures/engine_assembly.jpg}
    \caption{CAD model of the optical engine assembly showing the cylinder, cylinder head with intake runners (blue) and exhaust runners (red), and valve stems. The color coding distinguishes the intake side (blue) from the exhaust side (red), a convention maintained throughout the simulations.}
    \label{fig:engine_assembly}
\end{figure}

\section{CAD Model Development}

\subsection{Approach}

The CAD model for the optical engine was developed with several objectives:
\begin{enumerate}
    \item Accurately represent the combustion chamber geometry at all crank angles
    \item Provide a suitable description for the immersed boundary method
    \item Enable efficient mesh generation for both DNS and LES simulations
    \item Support validation against PIV measurements from the optical engine
\end{enumerate}

The model was created using engineering drawings and specifications from the optical engine facility. Key geometric features were measured and verified against the physical engine.

\subsection{Geometry Components}

The complete engine geometry includes the following components:

\subsubsection{Cylinder Head}

The cylinder head features a pentroof design typical of modern four-valve gasoline engines. The pentroof angle and roof shape strongly influence the tumble generation during the intake stroke. Key features include:
\begin{itemize}
    \item Pentroof angle: 15 degrees from horizontal
    \item Spark plug recess in the center of the roof
    \item Intake and exhaust valve seats
    \item Intake and exhaust port geometry (simplified)
\end{itemize}

\subsubsection{Piston}

The piston geometry includes:
\begin{itemize}
    \item Flat piston crown (as used in the optical engine)
    \item Valve relief cutouts to avoid valve-piston contact at high lift
    \item Piston ring land geometry (for near-wall studies)
\end{itemize}

The piston motion follows the standard crank-slider kinematics:
\begin{equation}
    x_p(\theta) = r\cos\theta + \sqrt{l^2 - r^2\sin^2\theta}
    \label{eq:piston_position}
\end{equation}
where $x_p$ is the piston position from the crankshaft center, $r$ is the crank radius (half of the stroke), $l$ is the connecting rod length, and $\theta$ is the crank angle.

\subsubsection{Valves}

The four-valve configuration includes:
\begin{itemize}
    \item Two intake valves (30 mm diameter)
    \item Two exhaust valves (26 mm diameter)
    \item Valve lift profiles as a function of crank angle
    \item Valve seat geometry for accurate port-cylinder interface
\end{itemize}

\begin{figure}[htbp]
    \centering
    \includegraphics[width=0.8\textwidth]{figures/engine_head_valves.jpg}
    \caption{Detail view of the cylinder head showing the valve seat geometry. The intake valve seats (blue rings) and exhaust valve seats (red rings) are critical regions where the flow accelerates through the narrow gap between valve and seat during opening and closing phases.}
    \label{fig:engine_head_valves}
\end{figure}

The valve seat geometry is particularly important for accurate flow prediction. During the valve opening phase, the flow accelerates through the narrow annular gap between the valve head and the seat, creating a high-velocity jet that enters the combustion chamber. The shape and angle of this seat directly influence the tumble and swirl generation.

\begin{figure}[htbp]
    \centering
    \includegraphics[width=0.6\textwidth]{figures/intake_port.jpg}
    \caption{Cross-sectional view of the intake port geometry showing the flow path from the intake runner to the combustion chamber. The port shape is designed to direct the incoming flow toward one side of the chamber to generate tumble motion.}
    \label{fig:intake_port}
\end{figure}

\subsection{STL Export for Immersed Boundary Method}

The CAD geometry is exported in STL (stereolithography) format for use with the immersed boundary method in PsiPhi. The STL format represents surfaces as a collection of triangular facets, each defined by three vertices and a normal vector.

\paragraph{Mesh Quality Considerations}

The quality of the STL mesh affects the accuracy of the immersed boundary representation:
\begin{itemize}
    \item \textbf{Resolution:} The triangular facets must be fine enough to resolve the surface curvature. A facet size of approximately 0.5 mm was used for curved surfaces.

    \item \textbf{Watertight mesh:} The surface must be closed (watertight) for the signed distance function to be computed correctly. All gaps and overlaps in the CAD model were repaired before export.

    \item \textbf{Normal orientation:} All triangle normals must point outward from the solid region. Inconsistent normals cause errors in the inside/outside classification.
\end{itemize}

\section{Simplified Channel Configuration}

Before applying the wall treatment to the full engine geometry, validation studies are performed in a simplified channel flow configuration. This section describes the channel geometry used for model development and verification.

\subsection{Channel Domain}

The channel flow domain is a rectangular box with:
\begin{itemize}
    \item Streamwise length: $L_x = 2\pi\delta$ (where $\delta$ is the channel half-height)
    \item Wall-normal extent: $L_y = 2\delta$
    \item Spanwise width: $L_z = \pi\delta$
\end{itemize}

This domain size is standard for channel flow DNS and LES, providing sufficient length for the largest turbulent structures to develop while remaining computationally tractable.

\subsection{Boundary Conditions}

The channel flow uses:
\begin{itemize}
    \item \textbf{Walls:} No-slip condition at $y = 0$ and $y = 2\delta$
    \item \textbf{Streamwise:} Periodic boundary conditions
    \item \textbf{Spanwise:} Periodic boundary conditions
    \item \textbf{Flow driving:} Constant pressure gradient to maintain target $Re_\tau$
\end{itemize}

\subsection{Grid Generation}

The Cartesian grid for the channel is straightforward:
\begin{itemize}
    \item Uniform spacing in streamwise and spanwise directions
    \item Option for stretching in the wall-normal direction to cluster points near the walls
    \item DNS grid: Typically $128 \times 128 \times 128$ or finer for $Re_\tau = 395$
    \item LES grid: $64 \times 64 \times 64$ or coarser, depending on wall treatment
\end{itemize}

The channel configuration serves as the primary validation case for the enhanced viscosity wall treatment. The simple geometry eliminates uncertainties related to complex boundaries, allowing a focused assessment of the wall model performance.

\section{Engine Mesh Generation}

\subsection{Challenges}

Mesh generation for the optical engine presents several challenges:

\begin{enumerate}
    \item \textbf{Moving boundaries:} The piston and valves move throughout the engine cycle, requiring either moving/deforming meshes or an immersed boundary approach.

    \item \textbf{Complex geometry:} The pentroof shape, valve seats, and piston cutouts create regions of high geometric complexity.

    \item \textbf{Varying clearances:} At top dead center (TDC), the clearance between piston and head can be less than 1 mm, requiring fine resolution.

    \item \textbf{Scale range:} The bore (84 mm) and smallest features (valve gaps) differ by two orders of magnitude.
\end{enumerate}

\subsection{Immersed Boundary Approach}

The immersed boundary method in PsiPhi addresses these challenges by using a fixed Cartesian background grid with the engine geometry represented through the signed distance function.

\paragraph{Advantages for Engine Simulation}

\begin{itemize}
    \item \textbf{No remeshing:} The background grid remains fixed while the geometry (represented by the signed distance function) moves through it.

    \item \textbf{Simple parallelization:} The structured Cartesian grid enables efficient domain decomposition and load balancing.

    \item \textbf{Consistent wall treatment:} The enhanced viscosity wall model is applied uniformly to all immersed surfaces---piston, head, and liner---without special treatment at different boundary types.

    \item \textbf{Automated distance computation:} The signed distance function is computed from the STL geometry at each time step, automatically handling the moving piston and valves.
\end{itemize}

\subsection{Grid Resolution}

For the engine simulations, the Cartesian grid resolution is chosen to balance computational cost with accuracy requirements:

\begin{itemize}
    \item \textbf{Base grid:} 256 cells across the bore diameter (cell size $\approx$ 0.33 mm)
    \item \textbf{Wall-normal resolution:} With the enhanced viscosity wall model, the first cell can be placed at $y^+ \approx 10$--30, relaxing the stringent requirements of wall-resolved LES
    \item \textbf{Total cell count:} Approximately 16 million cells for a full engine simulation
\end{itemize}

These grid requirements are feasible with current high-performance computing resources and allow multiple engine cycles to be simulated for statistical analysis.

\section{Summary}

This chapter has presented the CAD model development for the optical engine simulation:

\begin{itemize}
    \item The AVL 5811 optical engine at University of Duisburg-Essen provides a well-characterized research platform with extensive experimental data for validation.

    \item The CAD model accurately represents the combustion chamber geometry and is exported in STL format for the immersed boundary method.

    \item A simplified channel configuration is used for initial validation of the wall treatment before application to the full engine.

    \item The immersed boundary approach in PsiPhi enables efficient simulation of the moving engine geometry on a fixed Cartesian grid.
\end{itemize}

The next chapter presents the validation results from channel flow simulations, comparing the enhanced viscosity wall treatment against DNS data and analytical profiles.
